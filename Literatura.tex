
% I) bolji nacin: pomocu programa JabRef opisati svoje reference i pohraniti u datoteku ``Literatura''. U tekstu samo pozivati zeljene reference, a lista se sama formira.
%\bibliographystyle{IEEEtranHR}  % ``unsrt'', ``IEEEtran'', ``ieeetr''
% argument is your BibTeX string definitions and bibliography database(s)
%\bibliography{Literatura}


% II) rucno upisati svaku referencu redoslijedom kojim se prvi puta pozivaju u tekstu. Za ovo koristiti, ukloniti komentare ispred \begin i \end linije,  a broj \bibitem linija ovisi o kolicini referencirane literature

\begin{thebibliography}{99}
%
\bibitem{dobr:fiz1} Dobrinić: Fizika 1, Školska knjiga Zagreb; 2000.g.
%
\bibitem{antun} M.Joler: Predavanja; Riteh, 2021 - 2022
%
\bibitem{latex:web}  https://latex-tutorial.com/tutorials

\end{thebibliography}
